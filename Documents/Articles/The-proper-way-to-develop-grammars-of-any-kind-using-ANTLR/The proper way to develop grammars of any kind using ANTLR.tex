\documentclass[a4paper, 10pt, conference]{ieeeconf}

\usepackage{listings}
\PassOptionsToPackage{hyphens}{url}\usepackage{hyperref}

\title{\LARGE \bf
The proper way to develop grammars of any kind using ANTLR
}

\author{Ivan Kochurkin \\
Bauman Moscow State Technical University\\
{\tt\small ivan.kochurkin@gmail.com}}

\begin{document}

\maketitle
\thispagestyle{empty}
\pagestyle{empty}

\section{INTRODUCTION}

ANTLR is a parser generator that works on LL(*) grammars.  This parser generator already exists for over 20 years, its 4th version was released in 2013. Now it’s under development on GitHub. This tool allows users to create parsers in Java, C\#, Python2, Python3, and JavaScript runtimes. The support for C++, Swift, and Go could possibly be added soon.
Although ANTLR based parsers run fast and in a predictable manner and it’s easy enough to create grammars, there are certain challenges a developer of industrial or domain-specific language (DSL) faces. The present article highlights these issues. It also describes how to develop parsers for heterogeneous formal languages and presents a user friendly application for different grammars. Theoretical aspects of ANTLR grammars development are supported by practical implementation. The author has developed PHP and T-SQL grammars; he has significantly improved PL/SQL, Objective-C, and C\# grammars. These grammars are in great demand nowadays and used both in open-source and commercial software. PHP, PL/SQL, and T-SQL grammars are used in the signature-based code analyzer of PT Application Inspector, while Objective-C is used in codebeat, the web service for calculating code metrics\cite{codebeat-objectivec}. There are several issues and pull requests created in the official ANTLR4 grammars repo on GitHub \cite{antlr-grammars}. PHP grammar is used in the open-source tool for detecting malicious PHP scripts - mdetect\cite{mdetect}.

\section{Parsing problems}

\subsection{Ambiguity}

Natural language contains ambiguous expressions (like, "Flying planes can be dangerous"). Such constructions may also occur in a formal language. For instance, the statement in the following code fragment may be an expression statement or a function call statement depending on the situation.

\begin{lstlisting}
stat: expr ';' // expresssion
    | ID '(' ')' ';' // function call
    ;
expr: ID '(' ')'
    | INT
    ;
\end{lstlisting}
    
To solve the problem above, ANTLR uses semantic predicates or code insertions (written in the same language of the target runtime). Difficulties arise if you intend to write some universal predicates without duplicating grammar fragments. It has been shown that this problem is solvable.

\subsection{Handling whitespaces, comments}

During the parsing, we cannot simply remove the comments from the source code, as some important information (e.g., hard-coded passwords or misaligned strings) may be stored there. The disadvantage of parsing the comments, that were included into the grammar, makes the grammar overcomplicated, because each rule will contain a sub-rule with comments. ANTLR and other source code parsers (Roslyn) have different mechanisms for processing redundant strings\cite{antlr-roslyn}.

\subsection{Handling parse errors}

An important capability of each parser is error handling. Version 4 of ANTLR framework features the improved recovery of parsing process if syntax errors were detected. Still, sometimes it’s better to refine the grammar in order to achieve the best results for parsing errors. 

\section{Types of programming languages}

There is a large number of languages, in which the keywords are case-insensitive: Basic, Fortran, Pascal, etc. ANTLR tackles the problem of parsing such languages in two ways: by changing the grammar and by generating the additional runtime code.

There is a class of languages, which go through a preprocessing stage. In other words, the preprocessing of special directives is performed: some of these directives are used to check whether the existing code is compiled or not. These, for instance, are C-family languages (C\#, C++, and Objective-C) and PHP. The simultaneous parsing of the core code and directives is a difficult task, as it involves incorporating the processing logic into the grammar. However, in PHP, C\#, and Objective-C grammars this issue has been addressed in different ways, including the way specified above.

\section{Optimization of grammars and parsers}

Generally speaking, ANTLR parsers deliver high performance towards almost any code (a linear complexity over the input data \cite{ALL}), except in the case of a large number of adjacent homogeneous expressions. For example, the concatenation of a large number of strings. The grammar can be rewritten to handle deep recursion and, at the same time, remain fairly concise and maintain the same performance in all other cases.

Parallel and asynchronous processing is nowadays an essential part of both scientific and everyday calculations, even smartphone-assisted. The process of parsing using ANTLR can be parallelized, but to achieve maximum performance you should solve several issues related to the use of shared cache and memory consumption.

\section{The newer universal grammar editing tool}

There are several popular IDE plugins for developing and debugging ANTLR grammars: ANTLRWorks, ANTLR intellij-plugin-v4, and ANTLR Eclipse. However, all of them have certain disadvantages: they provide no support for developing grammars applicable to all runtimes without the IDE plugin. Desktop Antlr Grammar Editor (DAGE) \cite{DAGE}, an open source application, was developed to address these issues. It allows you to check the grammar thoroughly on the following four levels:

\begin{itemize}
\item Checking the syntax of the grammar.
\item Checking the semantics using ANTLR.
\item Checking for compilation errors in generated files.
\item Checking for parsing errors.
\end{itemize}

On the one hand, such decomposition of validation steps allows to achieve the optimal speed for the application, and provides the ability to find all the grammar errors with one single action on the other. The development implied solving some technical tasks, e.g. mapping of text files to display the relevant compilation error in the grammar file. In addition, one of the main objectives of DAGE was to address the problems of parsing, which were raised in this article. Thus, it supports case-insensitive languages and languages with preprocessing, and allows you to check the correctness of grammar applicable to all runtimes.

\section{CONCLUSION}

Currently, there are more than 100 ANTLR grammars in the official repo. However, some of these are applicable to a certain runtime only, mainly to Java. Future plans include bringing all the grammars to a common format and developing a web-version of Antlr Grammar Editor, so that any novice developer of the DSL parser (or any other industrial language) can quickly and easily start creating his own grammar right from his browser without knowing of additional dependencies. You can also share this grammar to other more advanced users, allowing them to immediately understand why the parser works improperly. 

\addtolength{\textheight}{-12cm}   % This command serves to balance the column lengths
                                  % on the last page of the document manually. It shortens
                                  % the textheight of the last page by a suitable amount.
                                  % This command does not take effect until the next page
                                  % so it should come on the page before the last. Make
                                  % sure that you do not shorten the textheight too much.

\begin{thebibliography}{99}

\bibitem{codebeat-objectivec} Objective-C support is dropping the BETA label, \url{https://hub.codebeat.co/blog/objective-c-support-is-dropping-the-beta-label}.

\bibitem{mdetect} Static-analysis detection of potentially malicious php code, \url{https://github.com/wsdookadr/mdetect}.

\bibitem{antlr-grammars} ANTLR grammars repository, \url{https://github.com/antlr/grammars-v4}.

\bibitem{antlr-roslyn} Ivan Kochurkin. Theory and Practice of Source Code Parsing with ANTLR and Roslyn, 2016, \url{http://blog.ptsecurity.com/2016/06/theory-and-practice-of-source-code.html}.

\bibitem{c2} Terence Parr. The Definitive ANTLR Reference. Pragmatic Bookshelf, 2013.

\bibitem{ALL} Terence Parr, Sam Harwell, Kathleen Fisher. Adaptive LL(*) Parsing: The Power of Dynamic Analysis. ACM New York, 2014:13.

\bibitem{DAGE} Desktop Antlr Grammar Editor, \url{https://github.com/KvanTTT/DAGE}

\end{thebibliography}

\end{document}
